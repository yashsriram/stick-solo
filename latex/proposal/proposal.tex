\documentclass[12pt]{article}
\usepackage[margin=1in]{geometry}
\usepackage{graphicx}
\usepackage{amsmath}
\usepackage{tikz}
\usepackage{hyperref}
\usepackage{enumitem}
\usepackage{cite}

\title{CSci 8980\\Project - Stick Solo\\Proposal}
\author{
Yashasvi Sriram Patkuri - patku001@umn.edu\\
}

\begin{document}
\maketitle

\paragraph{description}
The problem is to generate a controller for a given (human-like) stick-figure agent for the task of wall climbing (as in bouldering) so that the agent reaches the goal in a natural motion.
The environment and agent will be mostly in 2D.
The focus will be on using optimization based methods.

\paragraph{members}
\begin{enumerate}[nolistsep]
    \item Yashasvi Sriram Patkuri - patku001@umn.edu
\end{enumerate}

\paragraph{baseline}
Baseline is chosen as previous work on this problem, by me, using classical methods.
Demo: \url{https://www.youtube.com/watch?v=bZg6pS2gGPw}.
Github: \url{https://github.com/buggedbit/stick-solo}.

\paragraph{oracle}
There are two components of evaluation in this problem.
\begin{enumerate}[nolistsep]
    \item Reaching the goal.
    \item Moving naturally.
\end{enumerate}
While the first one can be evaluated automatically, the latter one is best evaluated using user-studies for now.
For example, if the controller produces motion for a climbing route which is not very distinguishable from some human climber's motion, that's a point.
An arbitrary threshold for number of such points can be set as one evaluation metric.

\paragraph{academic papers}
\cite{bull1995adaptive} proposes wall climbing robot gait control using genetic algorithms and Q-learning.
\cite{Grieco1998ASC} and \cite{351225} demonstrate wall-climbing robots.
\cite{kalisiak2001grasp} is one of the first papers to work on animating 2D parkour agents.
\cite{10.1145/3072959.3073707} addresses the problem of offline path and movement planning for wall climbing humanoid agents.
\cite{2017-TOG-deepLoco} discusses the idea of hierarchical controller generation for the tasks of walking, running, jogging. This is what I intend to apply to the task of wall climbing.

\paragraph{time estimates}
I think there are two major parts to the project.
Generating controller for local movements and higher level path planning.
Depending on how the first part evolves the time for second one might vary.
The most time in the first part should be taken by formulating dynamics, comparing and choosing RL algorithms, reward functions, implementation and debugging for incrementally sophisticated agents.

\bibliographystyle{apalike}
\bibliography{cites}

\end{document}
